\documentclass[a4paper, twoside]{article}

\setlength{\oddsidemargin}{0in} \setlength{\evensidemargin}{0in}
\setlength{\textwidth}{6.2in}
\setlength{\topmargin}{-0.3in} \setlength{\textheight}{9.8in}

\title{CS25710 - Travel Trouble}
\author{Tom Leaman (thl5)}

\begin{document}
\maketitle
\newpage
%\tableofcontents
%\newpage

\section{Hardware}

\subsection{Active Components}

The microcontroller will need to have sufficient I/O capabilities to communicate
with any extra sensing devices required, the ability to store the data as it's
captured and some way to transfer data to and from the device.

I have selected the Atmel ATSAM4SD32B which has 47 I/O pins for communicating
with an accelerometer, a GPS and a temperature sensor. It also has 2048 kb of
onboard flash memory (which should be sufficient to store the device's software
and cache data as it's being captured (which can then be transferred onto an SD
card via the onboard SD slot). Finally, it features a USB connector for
transferring data to and from the device.

The microcontroller has a small form factor (around 10cm x 10cm) and has a
low-power mode which I intend to activate when the device is not moving.

The device will require an accelerometer, I have selected an ADXL335 from Analog
Devices. This is a 6-pin analogue device (which will need to use the onboard A/D
converter on the microcontroller) and has a maximum shock value of 10kg which I
believe will be sufficient for the application.

GPS data will be captured by a GPS-634R, also from Analog Devices. This also
uses a 6-pin connector to interface with the Microcontroller and its onboard
Synchronous Serial Controller.

Temperature readings will be captured by a STMicroelectronics UDFN-4L, this is a
very small device requiring 2 connection pins for communication with the
Microcontroller. This device will be able to measure temperature accurate to
within a couple of degrees which, while not phenomenally accurate, I believe
should be enough accuracy to confirm or quash the client's suspicions about
temperature being a factor in the performance of the device.

\section{References}
\begin{itemize}
\item{Atmel ATSAM4SD32B -
	http://www.atmel.com/devices/SAM4SD32B.aspx?tab=overview}
\item{Analog Devices ADXL335 - http://www.nskelectronics.in/accelorometer.html}
\item{Analog Devices GPS-634R -
	http://www.nskelectronics.in/gps\_kit\_with\_patch\_antenna\_.html}
\item{STMicroelectronics UDFN-4L - http://www.st.com/web/catalog/sense\_power/FM89/SC294/PF129199}
\end{itemize}

\end{document}

