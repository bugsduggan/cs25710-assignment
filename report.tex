\documentclass[a4paper, twoside]{article}

\setlength{\oddsidemargin}{0in} \setlength{\evensidemargin}{0in}
\setlength{\textwidth}{6.2in}
\setlength{\topmargin}{-0.3in} \setlength{\textheight}{9.8in}

\title{CS25710 - Travel Trouble}
\author{Tom Leaman (thl5)}

\begin{document}
\maketitle
\newpage
%\tableofcontents
%\newpage

\section{Hardware}

\subsection{Components}

\subsubsection{Microcontroller}
% TODO add picture
I began by looking at boards specifically designed for prototyping this kind of
device, such as the Arduino. These devices are typically very well supported,
often with a strong community of maker types. I decided against actually using
an Arduino, however, as it is slightly larger than the microcontroller specified
below and also requires more power.

In the end, I have specified an Atmel SAM4SD32B. This microcontroller can run at
a lower voltage (1.62 - 3.6v) and is small enough (I hope) to be attached to the
inside of even the smallest suitcase without too much hinderance to the case's
primary function of storing one's luggage!

The SAM4SD32B is equipped with 47 general I/O pins, an A/D converter, a
Synchronous Serial Controller and USB and SD connections. The microcontroller is
also capable of running in a number of low-power modes which will be a necessity
in keeping the device's power consumption to a minimum during its operational
life.

\subsubsection{Accelerometer/Gyro}
% TODO add picture
I have selected a combined Accelerometer-Gyro, the MPU-6050. This is a 3-axis
Accelerometer and Gyro built into a single board. The device is capable of
measuring movement within a range of programmatically selected ranges which may
allow the client to record a range of movement events during the device's
deployment.

I did consider using a seperate accelerometer and gyro but these typically will
not only take more physical space and power but also, many of the 3-axis gyros I
found were actually more expensive on their own than the combined IMU specified
above.

\subsubsection{GPS}
% TODO add picture
When selecting a GPS, I tried to find one with a good signal strength (it will
likely be subject to a few layers of various materials which may hamper its
ability to communicate with the necessary satellites. I also tried to find
devices with an on-board antenna as this will reduce the overall form factor.

In the end, I have specified the EM-408. This is capable of 10m accuracy (which
I am hoping will be more than enough for this application), and a high degree of
sensitivity (-159dBm).

\subsubsection{Temperature}
% TODO add picture
I have specified the use of a TMP36 temperature sensor. This is a very small,
low-powered component which can measure temperature accurate to within 1 or 2
degrees (accuracy is reduced at higher temperatures).

\subsubsection{Cells}

\subsubsection{Other components}

\subsection{Connection diagram}

\subsection{Power requirements}

\section{References}
\begin{itemize}
	\item{foo}
\end{itemize}

\end{document}

